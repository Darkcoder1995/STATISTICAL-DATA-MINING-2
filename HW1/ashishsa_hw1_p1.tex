% Options for packages loaded elsewhere
\PassOptionsToPackage{unicode}{hyperref}
\PassOptionsToPackage{hyphens}{url}
%
\documentclass[
]{article}
\usepackage{lmodern}
\usepackage{amssymb,amsmath}
\usepackage{ifxetex,ifluatex}
\ifnum 0\ifxetex 1\fi\ifluatex 1\fi=0 % if pdftex
  \usepackage[T1]{fontenc}
  \usepackage[utf8]{inputenc}
  \usepackage{textcomp} % provide euro and other symbols
\else % if luatex or xetex
  \usepackage{unicode-math}
  \defaultfontfeatures{Scale=MatchLowercase}
  \defaultfontfeatures[\rmfamily]{Ligatures=TeX,Scale=1}
\fi
% Use upquote if available, for straight quotes in verbatim environments
\IfFileExists{upquote.sty}{\usepackage{upquote}}{}
\IfFileExists{microtype.sty}{% use microtype if available
  \usepackage[]{microtype}
  \UseMicrotypeSet[protrusion]{basicmath} % disable protrusion for tt fonts
}{}
\makeatletter
\@ifundefined{KOMAClassName}{% if non-KOMA class
  \IfFileExists{parskip.sty}{%
    \usepackage{parskip}
  }{% else
    \setlength{\parindent}{0pt}
    \setlength{\parskip}{6pt plus 2pt minus 1pt}}
}{% if KOMA class
  \KOMAoptions{parskip=half}}
\makeatother
\usepackage{xcolor}
\IfFileExists{xurl.sty}{\usepackage{xurl}}{} % add URL line breaks if available
\IfFileExists{bookmark.sty}{\usepackage{bookmark}}{\usepackage{hyperref}}
\hypersetup{
  pdftitle={ashishsa\_hw\_p1},
  hidelinks,
  pdfcreator={LaTeX via pandoc}}
\urlstyle{same} % disable monospaced font for URLs
\usepackage[margin=1in]{geometry}
\usepackage{color}
\usepackage{fancyvrb}
\newcommand{\VerbBar}{|}
\newcommand{\VERB}{\Verb[commandchars=\\\{\}]}
\DefineVerbatimEnvironment{Highlighting}{Verbatim}{commandchars=\\\{\}}
% Add ',fontsize=\small' for more characters per line
\usepackage{framed}
\definecolor{shadecolor}{RGB}{248,248,248}
\newenvironment{Shaded}{\begin{snugshade}}{\end{snugshade}}
\newcommand{\AlertTok}[1]{\textcolor[rgb]{0.94,0.16,0.16}{#1}}
\newcommand{\AnnotationTok}[1]{\textcolor[rgb]{0.56,0.35,0.01}{\textbf{\textit{#1}}}}
\newcommand{\AttributeTok}[1]{\textcolor[rgb]{0.77,0.63,0.00}{#1}}
\newcommand{\BaseNTok}[1]{\textcolor[rgb]{0.00,0.00,0.81}{#1}}
\newcommand{\BuiltInTok}[1]{#1}
\newcommand{\CharTok}[1]{\textcolor[rgb]{0.31,0.60,0.02}{#1}}
\newcommand{\CommentTok}[1]{\textcolor[rgb]{0.56,0.35,0.01}{\textit{#1}}}
\newcommand{\CommentVarTok}[1]{\textcolor[rgb]{0.56,0.35,0.01}{\textbf{\textit{#1}}}}
\newcommand{\ConstantTok}[1]{\textcolor[rgb]{0.00,0.00,0.00}{#1}}
\newcommand{\ControlFlowTok}[1]{\textcolor[rgb]{0.13,0.29,0.53}{\textbf{#1}}}
\newcommand{\DataTypeTok}[1]{\textcolor[rgb]{0.13,0.29,0.53}{#1}}
\newcommand{\DecValTok}[1]{\textcolor[rgb]{0.00,0.00,0.81}{#1}}
\newcommand{\DocumentationTok}[1]{\textcolor[rgb]{0.56,0.35,0.01}{\textbf{\textit{#1}}}}
\newcommand{\ErrorTok}[1]{\textcolor[rgb]{0.64,0.00,0.00}{\textbf{#1}}}
\newcommand{\ExtensionTok}[1]{#1}
\newcommand{\FloatTok}[1]{\textcolor[rgb]{0.00,0.00,0.81}{#1}}
\newcommand{\FunctionTok}[1]{\textcolor[rgb]{0.00,0.00,0.00}{#1}}
\newcommand{\ImportTok}[1]{#1}
\newcommand{\InformationTok}[1]{\textcolor[rgb]{0.56,0.35,0.01}{\textbf{\textit{#1}}}}
\newcommand{\KeywordTok}[1]{\textcolor[rgb]{0.13,0.29,0.53}{\textbf{#1}}}
\newcommand{\NormalTok}[1]{#1}
\newcommand{\OperatorTok}[1]{\textcolor[rgb]{0.81,0.36,0.00}{\textbf{#1}}}
\newcommand{\OtherTok}[1]{\textcolor[rgb]{0.56,0.35,0.01}{#1}}
\newcommand{\PreprocessorTok}[1]{\textcolor[rgb]{0.56,0.35,0.01}{\textit{#1}}}
\newcommand{\RegionMarkerTok}[1]{#1}
\newcommand{\SpecialCharTok}[1]{\textcolor[rgb]{0.00,0.00,0.00}{#1}}
\newcommand{\SpecialStringTok}[1]{\textcolor[rgb]{0.31,0.60,0.02}{#1}}
\newcommand{\StringTok}[1]{\textcolor[rgb]{0.31,0.60,0.02}{#1}}
\newcommand{\VariableTok}[1]{\textcolor[rgb]{0.00,0.00,0.00}{#1}}
\newcommand{\VerbatimStringTok}[1]{\textcolor[rgb]{0.31,0.60,0.02}{#1}}
\newcommand{\WarningTok}[1]{\textcolor[rgb]{0.56,0.35,0.01}{\textbf{\textit{#1}}}}
\usepackage{graphicx,grffile}
\makeatletter
\def\maxwidth{\ifdim\Gin@nat@width>\linewidth\linewidth\else\Gin@nat@width\fi}
\def\maxheight{\ifdim\Gin@nat@height>\textheight\textheight\else\Gin@nat@height\fi}
\makeatother
% Scale images if necessary, so that they will not overflow the page
% margins by default, and it is still possible to overwrite the defaults
% using explicit options in \includegraphics[width, height, ...]{}
\setkeys{Gin}{width=\maxwidth,height=\maxheight,keepaspectratio}
% Set default figure placement to htbp
\makeatletter
\def\fps@figure{htbp}
\makeatother
\setlength{\emergencystretch}{3em} % prevent overfull lines
\providecommand{\tightlist}{%
  \setlength{\itemsep}{0pt}\setlength{\parskip}{0pt}}
\setcounter{secnumdepth}{-\maxdimen} % remove section numbering

\title{ashishsa\_hw\_p1}
\author{}
\date{\vspace{-2.5em}}

\begin{document}
\maketitle

We initially consider the following Utility Matrix

\begin{Shaded}
\begin{Highlighting}[]
\KeywordTok{library}\NormalTok{(knitr)}
\NormalTok{a <-}\StringTok{ }\KeywordTok{c}\NormalTok{(}\DecValTok{4}\NormalTok{,}\OtherTok{NA}\NormalTok{,}\DecValTok{2}\NormalTok{)}
\NormalTok{b <-}\StringTok{ }\KeywordTok{c}\NormalTok{(}\DecValTok{5}\NormalTok{,}\DecValTok{3}\NormalTok{,}\OtherTok{NA}\NormalTok{)}
\NormalTok{c <-}\StringTok{ }\KeywordTok{c}\NormalTok{(}\OtherTok{NA}\NormalTok{,}\DecValTok{4}\NormalTok{,}\DecValTok{1}\NormalTok{)}
\NormalTok{d <-}\StringTok{ }\KeywordTok{c}\NormalTok{(}\DecValTok{5}\NormalTok{,}\DecValTok{3}\NormalTok{,}\DecValTok{3}\NormalTok{)}
\NormalTok{e <-}\StringTok{ }\KeywordTok{c}\NormalTok{(}\DecValTok{1}\NormalTok{,}\DecValTok{1}\NormalTok{,}\OtherTok{NA}\NormalTok{)}
\NormalTok{f <-}\StringTok{ }\KeywordTok{c}\NormalTok{(}\OtherTok{NA}\NormalTok{,}\DecValTok{2}\NormalTok{,}\DecValTok{4}\NormalTok{)}
\NormalTok{g <-}\StringTok{ }\KeywordTok{c}\NormalTok{(}\DecValTok{3}\NormalTok{,}\DecValTok{1}\NormalTok{,}\DecValTok{5}\NormalTok{)}
\NormalTok{h <-}\StringTok{ }\KeywordTok{c}\NormalTok{(}\DecValTok{2}\NormalTok{,}\OtherTok{NA}\NormalTok{,}\DecValTok{3}\NormalTok{)}
\NormalTok{df <-}\StringTok{ }\KeywordTok{data.frame}\NormalTok{(a,b,c,d,e,f,g,h)}
\end{Highlighting}
\end{Shaded}

Here we represent the missing values with NA.

\begin{Shaded}
\begin{Highlighting}[]
\NormalTok{df}
\end{Highlighting}
\end{Shaded}

\begin{verbatim}
##    a  b  c d  e  f g  h
## 1  4  5 NA 5  1 NA 3  2
## 2 NA  3  4 3  1  2 1 NA
## 3  2 NA  1 3 NA  4 5  3
\end{verbatim}

We now rename the rows in R.

\begin{Shaded}
\begin{Highlighting}[]
\KeywordTok{rownames}\NormalTok{(df) <-}\StringTok{ }\KeywordTok{c}\NormalTok{(}\StringTok{'A'}\NormalTok{,}\StringTok{'B'}\NormalTok{,}\StringTok{'C'}\NormalTok{)}
\NormalTok{df}
\end{Highlighting}
\end{Shaded}

\begin{verbatim}
##    a  b  c d  e  f g  h
## A  4  5 NA 5  1 NA 3  2
## B NA  3  4 3  1  2 1 NA
## C  2 NA  1 3 NA  4 5  3
\end{verbatim}

We now need to calculate the Jacard and cosine distance between users by
considering the utility matrix as Boolean. So Initially we convert the
utility matrix into Boolean

\begin{Shaded}
\begin{Highlighting}[]
\NormalTok{df[df}\OperatorTok{>}\DecValTok{0}\NormalTok{] <-}\StringTok{ }\DecValTok{1}
\NormalTok{df[df}\OperatorTok{<=}\DecValTok{0}\NormalTok{] <-}\StringTok{ }\DecValTok{0}
\NormalTok{df[}\KeywordTok{is.na}\NormalTok{(df)] <-}\StringTok{ }\DecValTok{0}
\NormalTok{df}
\end{Highlighting}
\end{Shaded}

\begin{verbatim}
##   a b c d e f g h
## A 1 1 0 1 1 0 1 1
## B 0 1 1 1 1 1 1 0
## C 1 0 1 1 0 1 1 1
\end{verbatim}

We have now converted the Utility Matrix into Binary Utility Matrix. Now
we calculate the Jackard Distance.

\#\#JACCARD DISTANCE

This can Manually be calculated as:

The Jaccard Index (between any two columns/users of the matrix M) is
a/a+b+c, where:

a = number of rows where both columns are 1

b = number of rows where the first row is 0 and the other row is 1

c = number of rows where the first row is 1 and the other row is 0

Now we calculate the Jaccard distance between (A,B) and (A,C):

Jaccard Distance between A,B:

a=4

b=2

c=2

J\_AB={[}4/(4+2+2){]}

J\_AB={[}1/2{]}

Jaccard Distance between A,C:

a=4

b=2

c=2

J\_AC={[}4/(4+2+2){]}

J\_AC={[}1/2{]}

\#\#COSINE DISTANCE

COSINE DISTANCE=1-COSINE SIMILARITY

COSINE
SIMILARITY=(A.B)/\textbar\textbar A\textbar\textbar.\textbar\textbar B\textbar\textbar{}

\#\#Cosine Distance between A,B:

A.B= (1\emph{0)+(1}1)+(0\emph{1)+(1}1)+(1\emph{1)+(0}1)+(1\emph{1)+(1}0)

A.B=(0+1+0+1+1+0+1+0)

A.B=4

\textbar\textbar A\textbar\textbar=(1\emph{1)+(1}1)+(0\emph{0)+(1}1)+(1\emph{1)+(0}0)+(1\emph{1)+(1}1)

\textbar\textbar A\textbar\textbar=6

\textbar\textbar B\textbar\textbar=(0\emph{0)+(1}1)+(1\emph{1)+(1}1)+(1\emph{1)+(1}1)+(1\emph{1)+(0}0)

\textbar\textbar B\textbar\textbar=6

COSINE SIMILARITY=4/(6*6)

COSINE SIMILARITY=4/36

COSINE DISTANCE=1-(4/36)

COSINE DISTANCE==32/36

\#\#Cosine Distance between A,C:

A.C= (1\emph{1)+(1}0)+(0\emph{1)+(1}1)+(1\emph{0)+(0}1)+(1\emph{1)+(1}1)

A.C=(1+0+0+1+0+0+1+1)

A.C=4

\textbar\textbar A\textbar\textbar=(1\emph{1)+(1}1)+(0\emph{0)+(1}1)+(1\emph{1)+(0}0)+(1\emph{1)+(1}1)

\textbar\textbar A\textbar\textbar=6

\textbar\textbar C\textbar\textbar=(1\emph{1)+(0}0)+(1\emph{1)+(1}1)+(0\emph{0)+(1}1)+(1\emph{1)+(1}1)

\textbar\textbar C\textbar\textbar=6

COSINE SIMILARITY=4/(6*6)

COSINE SIMILARITY=4/36

COSINE DISTANCE=1-(4/36)

COSINE DISTANCE==32/36

\#\#Now we consider that 3,4,5 are 1 and 1,2 are 0:

\begin{Shaded}
\begin{Highlighting}[]
\KeywordTok{library}\NormalTok{(knitr)}
\NormalTok{a <-}\StringTok{ }\KeywordTok{c}\NormalTok{(}\DecValTok{4}\NormalTok{,}\OtherTok{NA}\NormalTok{,}\DecValTok{2}\NormalTok{)}
\NormalTok{b <-}\StringTok{ }\KeywordTok{c}\NormalTok{(}\DecValTok{5}\NormalTok{,}\DecValTok{3}\NormalTok{,}\OtherTok{NA}\NormalTok{)}
\NormalTok{c <-}\StringTok{ }\KeywordTok{c}\NormalTok{(}\OtherTok{NA}\NormalTok{,}\DecValTok{4}\NormalTok{,}\DecValTok{1}\NormalTok{)}
\NormalTok{d <-}\StringTok{ }\KeywordTok{c}\NormalTok{(}\DecValTok{5}\NormalTok{,}\DecValTok{3}\NormalTok{,}\DecValTok{3}\NormalTok{)}
\NormalTok{e <-}\StringTok{ }\KeywordTok{c}\NormalTok{(}\DecValTok{1}\NormalTok{,}\DecValTok{1}\NormalTok{,}\OtherTok{NA}\NormalTok{)}
\NormalTok{f <-}\StringTok{ }\KeywordTok{c}\NormalTok{(}\OtherTok{NA}\NormalTok{,}\DecValTok{2}\NormalTok{,}\DecValTok{4}\NormalTok{)}
\NormalTok{g <-}\StringTok{ }\KeywordTok{c}\NormalTok{(}\DecValTok{3}\NormalTok{,}\DecValTok{1}\NormalTok{,}\DecValTok{5}\NormalTok{)}
\NormalTok{h <-}\StringTok{ }\KeywordTok{c}\NormalTok{(}\DecValTok{2}\NormalTok{,}\OtherTok{NA}\NormalTok{,}\DecValTok{3}\NormalTok{)}
\NormalTok{df <-}\StringTok{ }\KeywordTok{data.frame}\NormalTok{(a,b,c,d,e,f,g,h)}
\NormalTok{df[df}\OperatorTok{<=}\DecValTok{2}\NormalTok{] <-}\StringTok{ }\DecValTok{0}
\NormalTok{df[df}\OperatorTok{>=}\DecValTok{3}\NormalTok{] <-}\StringTok{ }\DecValTok{1}
\NormalTok{df[}\KeywordTok{is.na}\NormalTok{(df)] <-}\StringTok{ }\DecValTok{0}
\NormalTok{df}
\end{Highlighting}
\end{Shaded}

\begin{verbatim}
##   a b c d e f g h
## 1 1 1 0 1 0 0 1 0
## 2 0 1 1 1 0 0 0 0
## 3 0 0 0 1 0 1 1 1
\end{verbatim}

We have now converted the Utility Matrix into Binary Utility Matrix. Now
we calculate the Jackard Distance.

\#\#JACCARD DISTANCE

This can Manually be calculated as: The Jaccard Index (between any two
columns/users of the matrix M) is a/a+b+c, where:

a = number of rows where both columns are 1

b = number of rows where the first row is 0 and the other row is 1

c = number of rows where the first row is 1 and the other row is 0

Now we calculate the Jaccard distance between (A,B) and (A,C):

Jaccard Distance between A,B:

a=2

b=1

c=2

J\_AB={[}2/(2+1+2){]}

J\_AB={[}2/5{]}

Jaccard Distance between A,C:

a=2

b=2

c=2

J\_AC={[}2/(2+2+2){]}

J\_AC={[}2/6{]} \#\#COSINE DISTANCE

COSINE DISTANCE=1-COSINE SIMILARITY

COSINE
SIMILARITY=(A.B)/\textbar\textbar A\textbar\textbar.\textbar\textbar B\textbar\textbar{}

\#\#Cosine Distance between A,B: A.B=
(1\emph{0)+(1}1)+(0\emph{1)+(1}1)+(0\emph{0)+(0}0)+(1\emph{0)+(0}0)

A.B=(0+1+0+1+0+0+0+0)

A.B=2

\textbar\textbar A\textbar\textbar=(1\emph{1)+(1}1)+(0\emph{0)+(1}1)+(0\emph{0)+(0}0)+(1\emph{1)+(0}0)

\textbar\textbar A\textbar\textbar=4

\textbar\textbar B\textbar\textbar=(0\emph{0)+(1}1)+(1\emph{1)+(1}1)+(0\emph{0)+(0}0)+(0\emph{0)+(0}0)

\textbar\textbar B\textbar\textbar=3

COSINE SIMILARITY=2/(4*3)

COSINE SIMILARITY=2/12

COSINE DISTANCE=1-(2/12)

COSINE DISTANCE==10/12

\#\#Cosine Distance between A,C: A.C=
(1\emph{0)+(1}0)+(0\emph{0)+(1}1)+(0\emph{0)+(0}1)+(1\emph{1)+(0}1)

A.C=(0+0+0+1+0+0+1+0)

A.C=2

\textbar\textbar A\textbar\textbar=(1\emph{1)+(1}1)+(0\emph{0)+(1}1)+(0\emph{0)+(0}0)+(1\emph{1)+(0}0)

\textbar\textbar A\textbar\textbar=4

\textbar\textbar C\textbar\textbar=(0\emph{0)+(0}0)+(0\emph{0)+(1}1)+(0\emph{0)+(1}1)+(1\emph{1)+(1}1)

\textbar\textbar C\textbar\textbar=4

COSINE SIMILARITY=2/(4*4)

COSINE SIMILARITY=2/16

COSINE DISTANCE=1-(2/16)

COSINE DISTANCE==14/16

Normalize the matrix by subtracting from each nonblank entry the average
value for its user. Using this matrix, compute the cosine distance
between each pair of users.

We initially consider the following Utility Matrix

\begin{Shaded}
\begin{Highlighting}[]
\KeywordTok{library}\NormalTok{(knitr)}
\NormalTok{a <-}\StringTok{ }\KeywordTok{c}\NormalTok{(}\DecValTok{4}\NormalTok{,}\OtherTok{NA}\NormalTok{,}\DecValTok{2}\NormalTok{)}
\NormalTok{b <-}\StringTok{ }\KeywordTok{c}\NormalTok{(}\DecValTok{5}\NormalTok{,}\DecValTok{3}\NormalTok{,}\OtherTok{NA}\NormalTok{)}
\NormalTok{c <-}\StringTok{ }\KeywordTok{c}\NormalTok{(}\OtherTok{NA}\NormalTok{,}\DecValTok{4}\NormalTok{,}\DecValTok{1}\NormalTok{)}
\NormalTok{d <-}\StringTok{ }\KeywordTok{c}\NormalTok{(}\DecValTok{5}\NormalTok{,}\DecValTok{3}\NormalTok{,}\DecValTok{3}\NormalTok{)}
\NormalTok{e <-}\StringTok{ }\KeywordTok{c}\NormalTok{(}\DecValTok{1}\NormalTok{,}\DecValTok{1}\NormalTok{,}\OtherTok{NA}\NormalTok{)}
\NormalTok{f <-}\StringTok{ }\KeywordTok{c}\NormalTok{(}\OtherTok{NA}\NormalTok{,}\DecValTok{2}\NormalTok{,}\DecValTok{4}\NormalTok{)}
\NormalTok{g <-}\StringTok{ }\KeywordTok{c}\NormalTok{(}\DecValTok{3}\NormalTok{,}\DecValTok{1}\NormalTok{,}\DecValTok{5}\NormalTok{)}
\NormalTok{h <-}\StringTok{ }\KeywordTok{c}\NormalTok{(}\DecValTok{2}\NormalTok{,}\OtherTok{NA}\NormalTok{,}\DecValTok{3}\NormalTok{)}
\NormalTok{df <-}\StringTok{ }\KeywordTok{data.frame}\NormalTok{(a,b,c,d,e,f,g,h)}
\end{Highlighting}
\end{Shaded}

\begin{Shaded}
\begin{Highlighting}[]
\NormalTok{A <-}\StringTok{ }\KeywordTok{c}\NormalTok{(}\DecValTok{4}\NormalTok{,}\DecValTok{5}\NormalTok{,}\OtherTok{NA}\NormalTok{,}\DecValTok{5}\NormalTok{,}\DecValTok{1}\NormalTok{,}\OtherTok{NA}\NormalTok{,}\DecValTok{3}\NormalTok{,}\DecValTok{2}\NormalTok{)}
\NormalTok{B <-}\StringTok{ }\KeywordTok{c}\NormalTok{(}\OtherTok{NA}\NormalTok{,}\DecValTok{3}\NormalTok{,}\DecValTok{4}\NormalTok{,}\DecValTok{3}\NormalTok{,}\DecValTok{1}\NormalTok{,}\DecValTok{2}\NormalTok{,}\DecValTok{1}\NormalTok{,}\OtherTok{NA}\NormalTok{)}
\NormalTok{C <-}\StringTok{ }\KeywordTok{c}\NormalTok{(}\DecValTok{2}\NormalTok{,}\OtherTok{NA}\NormalTok{,}\DecValTok{1}\NormalTok{,}\DecValTok{3}\NormalTok{,}\OtherTok{NA}\NormalTok{,}\DecValTok{4}\NormalTok{,}\DecValTok{5}\NormalTok{,}\DecValTok{3}\NormalTok{)}

\NormalTok{mean_A <-}\StringTok{ }\KeywordTok{mean}\NormalTok{(A,}\DataTypeTok{na.rm =} \OtherTok{TRUE}\NormalTok{)}
\NormalTok{mean_A}
\end{Highlighting}
\end{Shaded}

\begin{verbatim}
## [1] 3.333333
\end{verbatim}

\begin{Shaded}
\begin{Highlighting}[]
\NormalTok{mean_B <-}\StringTok{ }\KeywordTok{mean}\NormalTok{(B,}\DataTypeTok{na.rm =} \OtherTok{TRUE}\NormalTok{)}
\NormalTok{mean_B}
\end{Highlighting}
\end{Shaded}

\begin{verbatim}
## [1] 2.333333
\end{verbatim}

\begin{Shaded}
\begin{Highlighting}[]
\NormalTok{mean_C <-}\StringTok{ }\KeywordTok{mean}\NormalTok{(C,}\DataTypeTok{na.rm =} \OtherTok{TRUE}\NormalTok{)}
\NormalTok{mean_C}
\end{Highlighting}
\end{Shaded}

\begin{verbatim}
## [1] 3
\end{verbatim}

Now substracting mean from every value in df we get:

\begin{Shaded}
\begin{Highlighting}[]
\NormalTok{A <-}\StringTok{ }\KeywordTok{c}\NormalTok{(}\DecValTok{1}\NormalTok{,}\DecValTok{2}\NormalTok{,}\OtherTok{NA}\NormalTok{,}\DecValTok{2}\NormalTok{,}\OperatorTok{-}\DecValTok{2}\NormalTok{,}\OtherTok{NA}\NormalTok{,}\DecValTok{0}\NormalTok{,}\OperatorTok{-}\DecValTok{1}\NormalTok{)}
\NormalTok{B <-}\StringTok{ }\KeywordTok{c}\NormalTok{(}\OtherTok{NA}\NormalTok{,}\DecValTok{1}\NormalTok{,}\DecValTok{2}\NormalTok{,}\DecValTok{1}\NormalTok{,}\OperatorTok{-}\DecValTok{1}\NormalTok{,}\DecValTok{0}\NormalTok{,}\OperatorTok{-}\DecValTok{1}\NormalTok{,}\OtherTok{NA}\NormalTok{)}
\NormalTok{C <-}\StringTok{ }\KeywordTok{c}\NormalTok{(}\OperatorTok{-}\DecValTok{1}\NormalTok{,}\OtherTok{NA}\NormalTok{,}\OperatorTok{-}\DecValTok{2}\NormalTok{,}\DecValTok{0}\NormalTok{,}\OtherTok{NA}\NormalTok{,}\DecValTok{1}\NormalTok{,}\DecValTok{2}\NormalTok{,}\DecValTok{0}\NormalTok{)}
\end{Highlighting}
\end{Shaded}

\#\#COSINE DISTANCE

COSINE DISTANCE=1-COSINE SIMILARITY

COSINE
SIMILARITY=(A.B)/\textbar\textbar A\textbar\textbar.\textbar\textbar B\textbar\textbar{}

we consider na as 0 \#\#Cosine Distance between A,B:

A.B=
(1\emph{0)+(2}1)+(0\emph{2)+(2}1)+(-2\emph{-1)+(0}0)+(0\emph{-1)+(-1}0)

A.B=(0+2+0+2+2+0+0+0)

A.B=6

\textbar\textbar A\textbar\textbar=(1\emph{1)+(2}2)+(0\emph{0)+(2}2)+(-2\emph{-2)+(0}0)+(0\emph{0)+(-1}-1)

\textbar\textbar A\textbar\textbar=14

\textbar\textbar B\textbar\textbar=(0\emph{0)+(1}1)+(2\emph{2)+(1}1)+(-1\emph{-1)+(0}0)+(-1\emph{-1)+(0}0)

\textbar\textbar B\textbar\textbar=8

COSINE SIMILARITY=6/(14*8)

COSINE SIMILARITY=2/112

COSINE DISTANCE=1-(2/112)

COSINE DISTANCE==110/112

\#\#Cosine Distance between A,C: A.C=
(1\emph{0)+(1}0)+(0\emph{0)+(1}1)+(0\emph{0)+(0}1)+(1\emph{1)+(0}1)

A.C=(0+0+0+1+0+0+1+0)

A.C=2

\textbar\textbar A\textbar\textbar=(1\emph{1)+(2}2)+(0\emph{0)+(2}2)+(-2\emph{-2)+(0}0)+(0\emph{0)+(-1}-1)

\textbar\textbar A\textbar\textbar=14

\textbar\textbar C\textbar\textbar=(-1\emph{-1)+(0}0)+(-2\emph{-2)+(0}0)+(0\emph{0)+(1}1)+(2\emph{2)+(0}0)

\textbar\textbar C\textbar\textbar=10

COSINE SIMILARITY=2/(14*10)

COSINE SIMILARITY=2/140

COSINE DISTANCE=1-(2/140)

COSINE DISTANCE==138/140

\end{document}
